\documentclass[]{article}
\usepackage{lmodern}
\usepackage{amssymb,amsmath}
\usepackage{ifxetex,ifluatex}
\usepackage{fixltx2e} % provides \textsubscript
\ifnum 0\ifxetex 1\fi\ifluatex 1\fi=0 % if pdftex
  \usepackage[T1]{fontenc}
  \usepackage[utf8]{inputenc}
\else % if luatex or xelatex
  \ifxetex
    \usepackage{mathspec}
  \else
    \usepackage{fontspec}
  \fi
  \defaultfontfeatures{Ligatures=TeX,Scale=MatchLowercase}
\fi
% use upquote if available, for straight quotes in verbatim environments
\IfFileExists{upquote.sty}{\usepackage{upquote}}{}
% use microtype if available
\IfFileExists{microtype.sty}{%
\usepackage{microtype}
\UseMicrotypeSet[protrusion]{basicmath} % disable protrusion for tt fonts
}{}
\usepackage[margin=1in]{geometry}
\usepackage{hyperref}
\hypersetup{unicode=true,
            pdftitle={ETC3250 2019 - Lab 1},
            pdfauthor={Dianne Cook},
            pdfborder={0 0 0},
            breaklinks=true}
\urlstyle{same}  % don't use monospace font for urls
\usepackage{color}
\usepackage{fancyvrb}
\newcommand{\VerbBar}{|}
\newcommand{\VERB}{\Verb[commandchars=\\\{\}]}
\DefineVerbatimEnvironment{Highlighting}{Verbatim}{commandchars=\\\{\}}
% Add ',fontsize=\small' for more characters per line
\usepackage{framed}
\definecolor{shadecolor}{RGB}{248,248,248}
\newenvironment{Shaded}{\begin{snugshade}}{\end{snugshade}}
\newcommand{\AlertTok}[1]{\textcolor[rgb]{0.94,0.16,0.16}{#1}}
\newcommand{\AnnotationTok}[1]{\textcolor[rgb]{0.56,0.35,0.01}{\textbf{\textit{#1}}}}
\newcommand{\AttributeTok}[1]{\textcolor[rgb]{0.77,0.63,0.00}{#1}}
\newcommand{\BaseNTok}[1]{\textcolor[rgb]{0.00,0.00,0.81}{#1}}
\newcommand{\BuiltInTok}[1]{#1}
\newcommand{\CharTok}[1]{\textcolor[rgb]{0.31,0.60,0.02}{#1}}
\newcommand{\CommentTok}[1]{\textcolor[rgb]{0.56,0.35,0.01}{\textit{#1}}}
\newcommand{\CommentVarTok}[1]{\textcolor[rgb]{0.56,0.35,0.01}{\textbf{\textit{#1}}}}
\newcommand{\ConstantTok}[1]{\textcolor[rgb]{0.00,0.00,0.00}{#1}}
\newcommand{\ControlFlowTok}[1]{\textcolor[rgb]{0.13,0.29,0.53}{\textbf{#1}}}
\newcommand{\DataTypeTok}[1]{\textcolor[rgb]{0.13,0.29,0.53}{#1}}
\newcommand{\DecValTok}[1]{\textcolor[rgb]{0.00,0.00,0.81}{#1}}
\newcommand{\DocumentationTok}[1]{\textcolor[rgb]{0.56,0.35,0.01}{\textbf{\textit{#1}}}}
\newcommand{\ErrorTok}[1]{\textcolor[rgb]{0.64,0.00,0.00}{\textbf{#1}}}
\newcommand{\ExtensionTok}[1]{#1}
\newcommand{\FloatTok}[1]{\textcolor[rgb]{0.00,0.00,0.81}{#1}}
\newcommand{\FunctionTok}[1]{\textcolor[rgb]{0.00,0.00,0.00}{#1}}
\newcommand{\ImportTok}[1]{#1}
\newcommand{\InformationTok}[1]{\textcolor[rgb]{0.56,0.35,0.01}{\textbf{\textit{#1}}}}
\newcommand{\KeywordTok}[1]{\textcolor[rgb]{0.13,0.29,0.53}{\textbf{#1}}}
\newcommand{\NormalTok}[1]{#1}
\newcommand{\OperatorTok}[1]{\textcolor[rgb]{0.81,0.36,0.00}{\textbf{#1}}}
\newcommand{\OtherTok}[1]{\textcolor[rgb]{0.56,0.35,0.01}{#1}}
\newcommand{\PreprocessorTok}[1]{\textcolor[rgb]{0.56,0.35,0.01}{\textit{#1}}}
\newcommand{\RegionMarkerTok}[1]{#1}
\newcommand{\SpecialCharTok}[1]{\textcolor[rgb]{0.00,0.00,0.00}{#1}}
\newcommand{\SpecialStringTok}[1]{\textcolor[rgb]{0.31,0.60,0.02}{#1}}
\newcommand{\StringTok}[1]{\textcolor[rgb]{0.31,0.60,0.02}{#1}}
\newcommand{\VariableTok}[1]{\textcolor[rgb]{0.00,0.00,0.00}{#1}}
\newcommand{\VerbatimStringTok}[1]{\textcolor[rgb]{0.31,0.60,0.02}{#1}}
\newcommand{\WarningTok}[1]{\textcolor[rgb]{0.56,0.35,0.01}{\textbf{\textit{#1}}}}
\usepackage{graphicx,grffile}
\makeatletter
\def\maxwidth{\ifdim\Gin@nat@width>\linewidth\linewidth\else\Gin@nat@width\fi}
\def\maxheight{\ifdim\Gin@nat@height>\textheight\textheight\else\Gin@nat@height\fi}
\makeatother
% Scale images if necessary, so that they will not overflow the page
% margins by default, and it is still possible to overwrite the defaults
% using explicit options in \includegraphics[width, height, ...]{}
\setkeys{Gin}{width=\maxwidth,height=\maxheight,keepaspectratio}
\IfFileExists{parskip.sty}{%
\usepackage{parskip}
}{% else
\setlength{\parindent}{0pt}
\setlength{\parskip}{6pt plus 2pt minus 1pt}
}
\setlength{\emergencystretch}{3em}  % prevent overfull lines
\providecommand{\tightlist}{%
  \setlength{\itemsep}{0pt}\setlength{\parskip}{0pt}}
\setcounter{secnumdepth}{0}
% Redefines (sub)paragraphs to behave more like sections
\ifx\paragraph\undefined\else
\let\oldparagraph\paragraph
\renewcommand{\paragraph}[1]{\oldparagraph{#1}\mbox{}}
\fi
\ifx\subparagraph\undefined\else
\let\oldsubparagraph\subparagraph
\renewcommand{\subparagraph}[1]{\oldsubparagraph{#1}\mbox{}}
\fi

%%% Use protect on footnotes to avoid problems with footnotes in titles
\let\rmarkdownfootnote\footnote%
\def\footnote{\protect\rmarkdownfootnote}

%%% Change title format to be more compact
\usepackage{titling}

% Create subtitle command for use in maketitle
\newcommand{\subtitle}[1]{
  \posttitle{
    \begin{center}\large#1\end{center}
    }
}

\setlength{\droptitle}{-2em}

  \title{ETC3250 2019 - Lab 1}
    \pretitle{\vspace{\droptitle}\centering\huge}
  \posttitle{\par}
    \author{Dianne Cook}
    \preauthor{\centering\large\emph}
  \postauthor{\par}
      \predate{\centering\large\emph}
  \postdate{\par}
    \date{March 4, 2019}


\begin{document}
\maketitle

Getting up and running with the computer:

\begin{itemize}
\tightlist
\item
  R and RStudio
\item
  RStudio Projects
\item
  RMarkdown
\item
  R syntax and basic functions
\end{itemize}

\hypertarget{what-is-r}{%
\subsection{What is R?}\label{what-is-r}}

\url{https://www.computerworld.com/article/2497143/business-intelligence/business-intelligence-beginner-s-guide-to-r-introduction.html}

From Wikipedia: ``R is a programming language and software environment
for statistical computing and graphics supported by the R Foundation for
Statistical Computing. The R language is widely used among statisticians
and data miners for developing statistical software and data analysis.''

R is free to use and has more than 14,000 (Feb 2019) user contributed
add-on packages on the Comprehensive R Archive Network (CRAN).

\hypertarget{what-is-rstudio}{%
\subsection{What is RStudio?}\label{what-is-rstudio}}

\href{http://jules32.github.io/resources/RStudio_intro/}{From Julie
Lowndes}:

If R were an airplane, RStudio would be the airport, providing many,
many supporting services that make it easier for you, the pilot, to take
off and go to awesome places. Sure, you can fly an airplane without an
airport, but having those runways and supporting infrastructure is a
game-changer.

The RStudio integrated development environment (IDE) has multiple
components including:

\begin{enumerate}
\def\labelenumi{\arabic{enumi}.}
\tightlist
\item
  Source editor (to edit your scripts):
\end{enumerate}

\begin{itemize}
\tightlist
\item
  Docking station for multiple files,
\item
  Useful shortcuts (``Knit''),
\item
  Highlighting/Tab-completion,
\item
  Code-checking (R, HTML, JS),
\item
  Debugging features\\
\end{itemize}

\begin{enumerate}
\def\labelenumi{\arabic{enumi}.}
\setcounter{enumi}{1}
\tightlist
\item
  Console window (to run your scripts, to test small pieces of code):
\end{enumerate}

\begin{itemize}
\tightlist
\item
  Highlighting/Tab-completion,
\item
  Search recent commands
\end{itemize}

\begin{enumerate}
\def\labelenumi{\arabic{enumi}.}
\setcounter{enumi}{2}
\tightlist
\item
  Other tabs/panes:
\end{enumerate}

\begin{itemize}
\tightlist
\item
  Graphics,
\item
  R documentation,
\item
  Environment pane,
\item
  File system navigation/access,
\item
  Tools for package development, git, etc
\end{itemize}

There's a cheatsheet in the ``Help'' menu, on tips for using this
interface.

\hypertarget{rstudio-projects}{%
\subsection{RStudio Projects}\label{rstudio-projects}}

\begin{itemize}
\tightlist
\item
  Project directories keep your work organized since you will keep your
  data, your code, your results all located in one place.
\item
  For the unit ETC3250, I have created a project on my laptop called
  \texttt{ETC3250}. Note that the name of the current project can be
  seen at the top right of the RStudio window.
\item
  YOU SHOULD ALWAYS WORK IN A PROJECT FOR THIS CLASS 😄
\end{itemize}

\begin{figure}
\centering
\includegraphics{projectname.png}
\caption{Using projects to organise your work}
\end{figure}

\begin{itemize}
\tightlist
\item
  .red{[}Each time you start RStudio{]} for this class, be sure to open
  the right project.
\end{itemize}

\hypertarget{exercise-1}{%
\subsection{Exercise 1}\label{exercise-1}}

Create a project for this unit, in the directory.

\begin{itemize}
\tightlist
\item
  File -\textgreater{} New Project -\textgreater{} Existing Directory
  -\textgreater{} Empty Project
\end{itemize}

\hypertarget{exercise-2}{%
\subsection{Exercise 2}\label{exercise-2}}

Open a new Rmarkdown document. You are going to want to call it
\texttt{Lab1} (it will automatically get the file extension
\texttt{.Rmd} when you save it).

\begin{itemize}
\tightlist
\item
  File -\textgreater{} New File -\textgreater{} R Markdown
  -\textgreater{} OK -\textgreater{} Knit HTML
\end{itemize}

\begin{figure}
\centering
\includegraphics{newFile.png}
\caption{Writing and computing with the one document}
\end{figure}

\hypertarget{what-is-rmarkdown}{%
\subsection{What is RMarkdown?}\label{what-is-rmarkdown}}

\begin{itemize}
\tightlist
\item
  R Markdown is an authoring format that enables easy creation of
  dynamic documents, presentations, and reports from R.
\item
  It combines the core syntax of \textbf{markdown} (an easy-to-write
  plain text format) \textbf{with embedded R code chunks} that are run
  so their output can be included in the final document.
\item
  R Markdown documents are fully reproducible (they can be automatically
  regenerated whenever underlying R code or data changes).
\end{itemize}

.red{[}There's a cheatsheet in the ``Help'' pages of RStudio on
Rmarkdown.{]}

When you click the \textbf{Knit} button a document will be generated
that includes both content as well as the output of any embedded R code
chunks within the document.

Equations can be included using LaTeX (\url{https://latex-project.org/})
commands like this:

\begin{verbatim}
$$s^2 = \frac{1}{n-1}\sum_{i=1}^n (x_i-\bar{x})^2.$$
\end{verbatim}

produce

\[s^2 = \frac{1}{n-1}\sum_{i=1}^n (x_i-\bar{x})^2.\]

We can also use inline mathematical symbols such as
\texttt{\$\textbackslash{}alpha\$} and
\texttt{\$\textbackslash{}infty\$}, which produce \(\alpha\) and
\(\infty\), respectively.

For more details on using R Markdown see
\url{http://rmarkdown.rstudio.com}. Spend a few minutes looking over
that website before continuing with this document.

\hypertarget{exercise-3}{%
\subsection{Exercise 3}\label{exercise-3}}

Look at the text in the \texttt{lab1.Rmd} document.

\begin{itemize}
\tightlist
\item
  What is R code?
\item
  How does \texttt{knitr} know that this is code to be run?
\item
  Using the RStudio IDE, work out how to run a chunk of code. Run this
  chunk, and then run the next chunk.
\item
  Using the RStudio IDE, how do you run just one line of R code?
\item
  Using the RStudio IDE, how do you highlight and run multiple lines of
  code?
\item
  What happens if you try to run a line that starts with `````\{r\}''?
  Or try to run a line of regular text from the document?
\item
  Using the RStudio IDE, \texttt{knit} the document into a Word
  document.
\end{itemize}

\hypertarget{some-r-basics}{%
\subsection{Some R Basics}\label{some-r-basics}}

\begin{itemize}
\tightlist
\item
  Type (into the console pane) and figure out what each of the following
  command is doing:
\end{itemize}

\begin{verbatim}
(100+2)/3
5*10^2
1/0
0/0
(0i-9)^(1/2)
sqrt(2*max(-10,0.2,4.5))+100
x <- sqrt(2*max(-10,0.2,4.5))+100
x
log(100)
log(100,base=10)
\end{verbatim}

\begin{itemize}
\tightlist
\item
  Check that these are equivalent: \texttt{y\ \textless{}-\ 100},
  \texttt{y\ =\ 100} and \texttt{100\ -\textgreater{}\ y}
\item
  R has rich support for documentation. Find the help page for the
  \texttt{mean} command, either from the help menu, or by typing one of
  these: \texttt{help(mean)} and \texttt{?mean}. Most help pages have
  examples at the bottom.
\item
  The \texttt{summary} command can be applied to almost anything to get
  a summary of the object. Try
  \texttt{summary(c(1,\ 3,\ 3,\ 4,\ 8,\ 8,\ 6,\ 7))}
\end{itemize}

\hypertarget{data-types}{%
\subsection{Data Types}\label{data-types}}

\begin{itemize}
\item
  \texttt{list}'s are heterogeneous (elements can have different types)
\item
  \texttt{data.frame}'s are heterogeneous but elements have same length
  (\texttt{dim} reports the dimensions and \texttt{colnames} shows the
  column names)
\item
  \texttt{vector}'s and \texttt{matrix}'s are homogeneous (elements have
  the same type), which would be why \texttt{c(1,\ "2")} ends up being a
  character string.
\item
  \texttt{function}'s can be written to save repeating code again and
  again
\item
  Try to understand these commands: \texttt{class}, \texttt{typeof},
  \texttt{is.numeric}, \texttt{is.vector} and \texttt{length}
\item
  See Hadley Wickham's online chapters on
  \href{http://adv-r.had.co.nz/Data-structures.html}{data structures
  (http://adv-r.had.co.nz/Data-structures.html)} for more
\end{itemize}

\hypertarget{operations}{%
\subsection{Operations}\label{operations}}

\begin{itemize}
\tightlist
\item
  Use built-in \emph{vectorized} functions to avoid loops
\end{itemize}

\begin{Shaded}
\begin{Highlighting}[]
\KeywordTok{set.seed}\NormalTok{(}\DecValTok{1000}\NormalTok{)}
\NormalTok{x <-}\StringTok{ }\KeywordTok{rnorm}\NormalTok{(}\DecValTok{6}\NormalTok{)}
\NormalTok{x}
\CommentTok{# [1] -0.44577826 -1.20585657  0.04112631  0.63938841 -0.78655436 -0.38548930}
\KeywordTok{sum}\NormalTok{(x }\OperatorTok{+}\StringTok{ }\DecValTok{10}\NormalTok{)}
\CommentTok{# [1] 57.85684}
\end{Highlighting}
\end{Shaded}

\hypertarget{section}{%
\subsection{}\label{section}}

\begin{itemize}
\tightlist
\item
  Use \texttt{{[}} to extract elements of a vector.
\end{itemize}

\begin{Shaded}
\begin{Highlighting}[]
\NormalTok{x[}\DecValTok{1}\NormalTok{]}
\CommentTok{# [1] -0.4457783}
\NormalTok{x[}\KeywordTok{c}\NormalTok{(T, F, T, T, F, F)]}
\CommentTok{# [1] -0.44577826  0.04112631  0.63938841}
\end{Highlighting}
\end{Shaded}

\hypertarget{section-1}{%
\subsection{}\label{section-1}}

\begin{itemize}
\tightlist
\item
  Extract \emph{named} elements with \texttt{\$}, \texttt{{[}{[}},
  and/or \texttt{{[}}
\end{itemize}

\begin{Shaded}
\begin{Highlighting}[]
\NormalTok{x <-}\StringTok{ }\KeywordTok{list}\NormalTok{(}
  \DataTypeTok{a =} \DecValTok{10}\NormalTok{,}
  \DataTypeTok{b =} \KeywordTok{c}\NormalTok{(}\DecValTok{1}\NormalTok{, }\StringTok{"2"}\NormalTok{)}
\NormalTok{)}
\NormalTok{x}\OperatorTok{$}\NormalTok{a}
\CommentTok{# [1] 10}
\NormalTok{x[[}\StringTok{"a"}\NormalTok{]]}
\CommentTok{# [1] 10}
\NormalTok{x[}\StringTok{"a"}\NormalTok{]}
\CommentTok{# $a}
\CommentTok{# [1] 10}
\end{Highlighting}
\end{Shaded}

\hypertarget{examining-structure}{%
\subsection{Examining `structure'}\label{examining-structure}}

\begin{itemize}
\tightlist
\item
  \texttt{str()} is a very useful \texttt{R} function. It shows you the
  ``structure'' of (almost) \emph{any} R object (and \emph{everything}
  in R is an object!!!)
\end{itemize}

\begin{Shaded}
\begin{Highlighting}[]
\KeywordTok{str}\NormalTok{(x)}
\CommentTok{# List of 2}
\CommentTok{#  $ a: num 10}
\CommentTok{#  $ b: chr [1:2] "1" "2"}
\end{Highlighting}
\end{Shaded}

\hypertarget{missing-values}{%
\subsection{Missing Values}\label{missing-values}}

\begin{itemize}
\tightlist
\item
  \texttt{NA} is the indicator of a missing value in R
\item
  Most functions have options for handling missings
\end{itemize}

\begin{Shaded}
\begin{Highlighting}[]
\NormalTok{x <-}\StringTok{ }\KeywordTok{c}\NormalTok{(}\DecValTok{50}\NormalTok{, }\DecValTok{12}\NormalTok{, }\OtherTok{NA}\NormalTok{, }\DecValTok{20}\NormalTok{)}
\KeywordTok{mean}\NormalTok{(x)}
\CommentTok{# [1] NA}
\KeywordTok{mean}\NormalTok{(x, }\DataTypeTok{na.rm=}\OtherTok{TRUE}\NormalTok{)}
\CommentTok{# [1] 27.33333}
\end{Highlighting}
\end{Shaded}

\hypertarget{counting-categories}{%
\subsection{Counting Categories}\label{counting-categories}}

\begin{itemize}
\tightlist
\item
  the \texttt{table} function can be used to tabulate numbers
\end{itemize}

\begin{Shaded}
\begin{Highlighting}[]
\KeywordTok{table}\NormalTok{(}\KeywordTok{c}\NormalTok{(}\DecValTok{1}\NormalTok{, }\DecValTok{2}\NormalTok{, }\DecValTok{3}\NormalTok{, }\DecValTok{1}\NormalTok{, }\DecValTok{2}\NormalTok{, }\DecValTok{8}\NormalTok{, }\DecValTok{1}\NormalTok{, }\DecValTok{4}\NormalTok{, }\DecValTok{2}\NormalTok{))}
\CommentTok{# }
\CommentTok{# 1 2 3 4 8 }
\CommentTok{# 3 3 1 1 1}
\end{Highlighting}
\end{Shaded}

\hypertarget{functions}{%
\subsection{Functions}\label{functions}}

One of the powerful aspects of R is to build on the reproducibility. If
you are going to do the same analysis over and over again, compile these
operations into a function that you can then apply to different data
sets.

\begin{Shaded}
\begin{Highlighting}[]
\NormalTok{average <-}\StringTok{ }\ControlFlowTok{function}\NormalTok{(x)}
\NormalTok{\{}
  \KeywordTok{return}\NormalTok{(}\KeywordTok{sum}\NormalTok{(x)}\OperatorTok{/}\KeywordTok{length}\NormalTok{(x))}
\NormalTok{\}}

\NormalTok{y1 <-}\StringTok{ }\KeywordTok{c}\NormalTok{(}\DecValTok{1}\NormalTok{,}\DecValTok{2}\NormalTok{,}\DecValTok{3}\NormalTok{,}\DecValTok{4}\NormalTok{,}\DecValTok{5}\NormalTok{,}\DecValTok{6}\NormalTok{)}
\KeywordTok{average}\NormalTok{(y1)}
\CommentTok{# [1] 3.5}

\NormalTok{y2 <-}\StringTok{ }\KeywordTok{c}\NormalTok{(}\DecValTok{1}\NormalTok{, }\DecValTok{9}\NormalTok{, }\DecValTok{4}\NormalTok{, }\DecValTok{4}\NormalTok{, }\DecValTok{0}\NormalTok{, }\DecValTok{1}\NormalTok{, }\DecValTok{15}\NormalTok{)}
\KeywordTok{average}\NormalTok{(y2)}
\CommentTok{# [1] 4.857143}
\end{Highlighting}
\end{Shaded}

Now write a function to compute the mode of some vector, and confirm
that it returns \texttt{4} when applied on
\texttt{y\ \textless{}-\ c(1,\ 1,\ 2,\ 4,\ 4,\ 4,\ 9,\ 4,\ 4,\ 8)}

\hypertarget{exercise-4}{%
\subsection{Exercise 4}\label{exercise-4}}

\begin{itemize}
\tightlist
\item
  What's an R \texttt{package}?
\item
  How do you install a package?
\item
  How does the \texttt{library()} function relates to a
  \texttt{package}?
\item
  How often do you load a \texttt{package}?
\item
  Install and load the package \texttt{ISLR}
\end{itemize}

\hypertarget{getting-data}{%
\subsection{Getting data}\label{getting-data}}

Data can be found in R packages

\begin{Shaded}
\begin{Highlighting}[]
\KeywordTok{library}\NormalTok{(tidyverse)}
\KeywordTok{data}\NormalTok{(economics, }\DataTypeTok{package =} \StringTok{"ggplot2"}\NormalTok{)}
\CommentTok{# data frames are essentially a list of vectors}
\KeywordTok{glimpse}\NormalTok{(economics)}
\CommentTok{# Observations: 574}
\CommentTok{# Variables: 6}
\CommentTok{# $ date     <date> 1967-07-01, 1967-08-01, 1967-09-01, 1967-10-01, 1967...}
\CommentTok{# $ pce      <dbl> 507.4, 510.5, 516.3, 512.9, 518.1, 525.8, 531.5, 534....}
\CommentTok{# $ pop      <int> 198712, 198911, 199113, 199311, 199498, 199657, 19980...}
\CommentTok{# $ psavert  <dbl> 12.5, 12.5, 11.7, 12.5, 12.5, 12.1, 11.7, 12.2, 11.6,...}
\CommentTok{# $ uempmed  <dbl> 4.5, 4.7, 4.6, 4.9, 4.7, 4.8, 5.1, 4.5, 4.1, 4.6, 4.4...}
\CommentTok{# $ unemploy <int> 2944, 2945, 2958, 3143, 3066, 3018, 2878, 3001, 2877,...}
\end{Highlighting}
\end{Shaded}

These are not usually kept up to date but are good for practicing your
analysis skills on.

Or in their own packages

\begin{Shaded}
\begin{Highlighting}[]
\KeywordTok{library}\NormalTok{(gapminder)}
\KeywordTok{glimpse}\NormalTok{(gapminder)}
\CommentTok{# Observations: 1,704}
\CommentTok{# Variables: 6}
\CommentTok{# $ country   <fct> Afghanistan, Afghanistan, Afghanistan, Afghanistan, ...}
\CommentTok{# $ continent <fct> Asia, Asia, Asia, Asia, Asia, Asia, Asia, Asia, Asia...}
\CommentTok{# $ year      <int> 1952, 1957, 1962, 1967, 1972, 1977, 1982, 1987, 1992...}
\CommentTok{# $ lifeExp   <dbl> 28.801, 30.332, 31.997, 34.020, 36.088, 38.438, 39.8...}
\CommentTok{# $ pop       <int> 8425333, 9240934, 10267083, 11537966, 13079460, 1488...}
\CommentTok{# $ gdpPercap <dbl> 779.4453, 820.8530, 853.1007, 836.1971, 739.9811, 78...}
\end{Highlighting}
\end{Shaded}

I primarily use the \texttt{readr} package (part of the
\texttt{tidyverse} suite) for reading data now. It mimics the base R
reading functions but is implemented in \texttt{C} so reads large files
quickly, and it also attempts to identify the types of variables.

\begin{Shaded}
\begin{Highlighting}[]
\NormalTok{candy <-}\StringTok{ }\KeywordTok{read_csv}\NormalTok{(}\StringTok{"https://raw.githubusercontent.com/fivethirtyeight/data/master/candy-power-ranking/candy-data.csv"}\NormalTok{)}
\KeywordTok{glimpse}\NormalTok{(candy)}
\CommentTok{# Observations: 85}
\CommentTok{# Variables: 13}
\CommentTok{# $ competitorname   <chr> "100 Grand", "3 Musketeers", "One dime", "One...}
\CommentTok{# $ chocolate        <dbl> 1, 1, 0, 0, 0, 1, 1, 0, 0, 0, 1, 0, 0, 0, 0, ...}
\CommentTok{# $ fruity           <dbl> 0, 0, 0, 0, 1, 0, 0, 0, 0, 1, 0, 1, 1, 1, 1, ...}
\CommentTok{# $ caramel          <dbl> 1, 0, 0, 0, 0, 0, 1, 0, 0, 1, 0, 0, 0, 0, 0, ...}
\CommentTok{# $ peanutyalmondy   <dbl> 0, 0, 0, 0, 0, 1, 1, 1, 0, 0, 0, 0, 0, 0, 0, ...}
\CommentTok{# $ nougat           <dbl> 0, 1, 0, 0, 0, 0, 1, 0, 0, 0, 1, 0, 0, 0, 0, ...}
\CommentTok{# $ crispedricewafer <dbl> 1, 0, 0, 0, 0, 0, 0, 0, 0, 0, 0, 0, 0, 0, 0, ...}
\CommentTok{# $ hard             <dbl> 0, 0, 0, 0, 0, 0, 0, 0, 0, 0, 0, 0, 0, 0, 1, ...}
\CommentTok{# $ bar              <dbl> 1, 1, 0, 0, 0, 1, 1, 0, 0, 0, 1, 0, 0, 0, 0, ...}
\CommentTok{# $ pluribus         <dbl> 0, 0, 0, 0, 0, 0, 0, 1, 1, 0, 0, 1, 1, 1, 0, ...}
\CommentTok{# $ sugarpercent     <dbl> 0.732, 0.604, 0.011, 0.011, 0.906, 0.465, 0.6...}
\CommentTok{# $ pricepercent     <dbl> 0.860, 0.511, 0.116, 0.511, 0.511, 0.767, 0.7...}
\CommentTok{# $ winpercent       <dbl> 66.97173, 67.60294, 32.26109, 46.11650, 52.34...}
\end{Highlighting}
\end{Shaded}

You can pull data together yourself, or look at data compiled by someone
else.

\hypertarget{question-1}{%
\subsection{Question 1}\label{question-1}}

\begin{itemize}
\item
  Look at the \texttt{economics} data in the \texttt{ggplot2} package.
  Can you think of two questions you could answer using these variables?
\item
  Write these into your \texttt{.Rmd} file.
\end{itemize}

\hypertarget{question-2}{%
\subsection{Question 2}\label{question-2}}

\begin{itemize}
\item
  Read the documentation for \texttt{gapminder} data. Can you think of
  two questions you could answer using these variables?
\item
  Write these into your \texttt{.Rmd} file.
\end{itemize}

\hypertarget{question-3}{%
\subsection{Question 3}\label{question-3}}

\begin{itemize}
\item
  Read the documentation for \texttt{pedestrian\ sensor} data. Can you
  think of two questions you could answer using these variables?
\item
  Write these into your \texttt{.Rmd} file.
\end{itemize}

\hypertarget{question-4}{%
\subsection{Question 4}\label{question-4}}

\begin{enumerate}
\def\labelenumi{\arabic{enumi}.}
\tightlist
\item
  Read in the OECD PISA data (file \texttt{student.rds} is available at
  from the course web site)
\item
  Tabulate the countries (CNT)
\item
  Extract the values for Australia (AUS) and Shanghai (QCN)
\item
  Compute the average and standard deviation of the reading scores
  (PV1READ), for each country
\item
  Write a few sentences explaining what you learn about reading in these
  two countries.
\end{enumerate}

\hypertarget{homework}{%
\subsection{Homework}\label{homework}}

Using your \emph{free} DataCamp account, work your way through the free
tutorial
\href{https://www.datacamp.com/courses/free-introduction-to-r}{Introduction
to R}. This provides some good insights on the data types you will
commonly use in R.

\hypertarget{got-a-question}{%
\subsection{Got a question?}\label{got-a-question}}

It is always good to try to solve your problem yourself first. Most
likely the error is a simple one, like a missing ``)'' or ``,''. For
deeper questions about packages, analyses and functions, making your Rmd
into a document, or simply the error that is being generated, you can
often google for an answer. Often, you will be directed to
\href{http://stackoverflow.com}{Q/A site: http://stackoverflow.com}.

Stackoverflow is a great place to get answers to tougher questions about
R and also data analysis. You always need to check that someone hasn't
asked it before, the answer might already be available for you. If not,
make a
\href{https://reprex.tidyverse.org/articles/reprex-dos-and-donts.html}{reproducible
example of your problem, following the guidelines here} and ask away.
Remember these people that kindly answer questions on stackoverflow have
day jobs too, and do this community support as a kindness to all of us.


\end{document}
